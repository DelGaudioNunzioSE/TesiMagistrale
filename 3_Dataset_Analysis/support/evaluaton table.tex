% Riga fissa con campo modificabile
\newcommand{\rowMetric}[2]{#1 & #2 \\ \hline}


\newcommand{\sectionHeaderCentered}[1]{%
  \addlinespace[1ex]
  \multicolumn{2}{c}{\textbf{#1}} \\
  \addlinespace[0.5ex]\midrule
}

% Tabella principale: usa un comando esterno per riempire la seconda colonna
\newcommand{\evaluationTable}[1]{
\begin{table}[H]
\centering
\begin{tabularx}{\linewidth}{|p{7.5cm}|X|}

\hline
\sectionHeaderCentered{Total number of code samples available}
\rowMetric{(a) Total number of human code}{\csname #1HumanCode\endcsname}
\rowMetric{(b) Total number of LLMs code}{\csname #1LLMCode\endcsname}

\sectionHeaderCentered{LLMs used to generate synthetic code}
\rowMetric{(a) Number of LLMs}{\csname #1NumLLMs\endcsname}
\rowMetric{(b) Diversity among LLMs}{\csname #1LLMDiversity\endcsname}
\rowMetric{(c) Actual degree of use in contemporary times}{\csname #1CurrentUse\endcsname}

\sectionHeaderCentered{Code diversity}
\rowMetric{(a) Different programming languages}{\csname #1Languages\endcsname}
\rowMetric{(b) Different types of code}{\csname #1CodeTypes\endcsname}
\rowMetric{(c) Code size}{\csname #1CodeSize\endcsname}
\rowMetric{(d) Code context}{\csname #1CodeContext\endcsname}

\sectionHeaderCentered{Validity information}
\rowMetric{(a) Generation prompts}{\csname #1Prompts\endcsname}
\rowMetric{(b) Source of human-written code}{\csname #1Sources\endcsname}
\rowMetric{(c) Code quality}{\csname #1CodeQuality\endcsname}
\rowMetric{(d) Paper reliability perception}{\csname #1Reliability\endcsname}

\hline
\end{tabularx}
\caption{Evaluation Summary: #1 Dataset}
\end{table}
}
