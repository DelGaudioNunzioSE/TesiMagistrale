\subsection{Pan et al}
\label{section:Pan et al}
The main objective of this paper is to demonstrate, 
beyond any doubt, that detection methods commonly used for 
natural language are not reliable when applied to code. 
To achieve this, the authors construct a dedicated dataset.


\begin{itemize}
    \item \textbf{GPT}: In the paper is not specified what gpt.
\end{itemize}

%%%%
%%%%

\subsubsection*{Strengths}
\begin{itemize}
    \item Generate code by different Prompt.
    \begin{itemize}
        \item Prompt explaining the problem and asking for a solution;
        \item Removal of stop words from the prompt;
        \item Request to imitate a human;
        \item Request to generate code without any comments;
        \item Request to also generate testing code;
        \item Solution with generic test cases not specifically tailored to the generated code;
        \item Solution using the unfittest framework;
        \item Explicit request for a long code;
        \item Explicit request for a short code.
    \end{itemize}
    \item Use different paraphrasing method.
    \begin{itemize}
        \item Replacement of variable names;
        \item Replacement of function names;
        \item Replacement of both variable and function names;
        \item Addition of unnecessary code to the solution.
    \end{itemize}
    \item The source problem is specified in the dataset.
\end{itemize}


%%%%
%%%%
\subsubsection*{Weaknesses}
\begin{itemize}
    \item In the paper is not specified what gpt is used in order to generate code.
    \item Only competitive code.
\end{itemize}


\subsubsection*{Code Quality}
In order to preserve quality in the LLMs code source:
\begin{itemize}
    \item \textit{”manual valiation to ensure the correctness of the dataset”}
\end{itemize}


%%%%
%%%%
\subsubsection*{Final evaluation}


\expandafter\def\csname Pan-et-alHumanCode\endcsname{5,069}
\expandafter\def\csname Pan-et-alLLMCode\endcsname{65,897}
\expandafter\def\csname Pan-et-alNumLLMs\endcsname{1 \textit{GPT}}
\expandafter\def\csname Pan-et-alLLMDiversity\endcsname{1}
\expandafter\def\csname Pan-et-alCurrentUse\endcsname{unspecified}
\expandafter\def\csname Pan-et-alLanguages\endcsname{Python}
\expandafter\def\csname Pan-et-alCodeTypes\endcsname{unspecified}
\expandafter\def\csname Pan-et-alCodeSize\endcsname{1\textsuperscript{st} percentile: 54 words,\newline 3\textsuperscript{rd} percentile: 83 words}
\expandafter\def\csname Pan-et-alCodeContext\endcsname{competitive}
\expandafter\def\csname Pan-et-alPrompts\endcsname{provided}
\expandafter\def\csname Pan-et-alSources\endcsname{ Quescol \cite{quescol2023}, Kaggle \cite{wikipediaKaggle2023}}
\expandafter\def\csname Pan-et-alCodeQuality\endcsname{filtered by human}
\expandafter\def\csname Pan-et-alReliability\endcsname{Peer-review paper}


\evaluationTable{Pan-et-al}
Despite the reliability of this work, given that it is a peer-reviewed paper, 
the dataset consists of only a single programming language and a single LLM 
(which is known to be GPT, though the specific version is not disclosed). 
Moreover, it is not explicitly stated whether the generated code is actually 
functional.

