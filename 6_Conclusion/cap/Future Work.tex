\clearpage
\section{Future Work}
There are many other methods proposed in the literature 
that could not be tested because their work was not available. 
It would be worth considering the possibility of retrieving these 
implementations or developing and testing them. Additionally, it 
would be a good initiative to make the interface more easily 
modifiable by any user, making it simple for anyone to add datasets 
or detection methods. The goal would be to encourage the community 
to make detection code publicly available. Perhaps this interface 
could be hosted on a platform to allow any user to perform detection 
tasks without the need for local hardware capable of running the 
recommended detection method.

It would be an interesting task to extend the CodeMirage work by 
adding competitive code as well. This would be useful because this 
thesis suggests that the most solid methods rely on transformer-encoders 
fine-tuned for the detection task. Therefore, it is clear that having 
a large dataset with all types of code, languages, and as many LLMs as 
possible can only benefit any kind of training.

More variants of those tested in this work could be explored, 
such as using a more complex classification head for both BiScope 
and GPTSniffer/CodeT5+. Detection methods based on machine learning 
could also be tested with different techniques, hyperparameters, optimizers, etc.