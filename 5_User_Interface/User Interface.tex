\chapter{User Interface}
The user interface was initially designed to resemble any typical 
detector interface, such as GPTZero. The original idea was simply 
to provide a text box where the code to be analysed could be inserted, 
along with a graph to visualize the probability that the code was 
generated by an LLM. However, during the evaluation of the methods, 
it became clear that an effective code detection method does not 
currently exist. For this reason, the project evolved to not only 
allow a user to evaluate a single piece of code, as originally 
intended, but also to enable the retrieval of datasets and the 
testing of methods presented in this work. The motivation behind 
this decision was to avoid repeating the mistakes of some published 
works in this field and to make it extremely simple for any user to 
test and replicate the results obtained.


\begin{figure}[H]
    \centering
    \includegraphics[width=0.8\linewidth]{img/gptzero.png}
    \caption{gptzero interface}
    \label{fig:gptzero}
\end{figure}
\clearpage
\section{Requirements Analysis}
Unfortunately, it was not possible to distribute questionnaires regarding 
the requirements for a code detection interface. However, it wasn't difficult 
to find successful portals to draw inspiration from. Methods for detecting 
natural text on the web tend to have a clean interface, display a probability 
percentage of detection, and show the sections of the text that most likely 
indicate that the text is generated by an LLM. This last request was not 
feasible due to the different detection method used for code 
(code embedding does not allow us to understand which sections of 
the code lead to the conclusion that it was generated by an LLM).

A requirement set during development was the need to create a web-based 
interface. In this way, it would not be complicated, if desired, to make 
the code detector easily accessible to any user via the web by setting up 
an appropriately equipped server. Given the request to make this interface 
not only a simple detector but also a validator of the methods presented 
and datasets proposed, it was crucial to make the detection interface easily 
and immediately accessible, clearly separated from other sections.

Given the large number of datasets and methods presented, it was necessary 
to visually differentiate each dedicated page, using colors (for datasets) 
and different page layouts for the detection methods. In fact, each detection 
method had to provide several possible operations, such as: training a model or 
loading the weights of a previously trained model, easily launching training, 
allowing methods based on perplexity to calculate the best threshold or set it 
manually before launching any test. Additionally, for each dataset, an automated 
dataset balancing method, an automatic split for training, validation, and test 
sets, and a standard format for each dataset had to be provided.
\clearpage
\section{Design Choices}
It was decided to design the text box to recognize programming 
language patterns and allow the code to be displayed in a 
user-friendly manner according to the programming language. 
For this reason, a section was added to select the 
language of the code. The language selection does not affect 
the detection in any way (at least in the current method). 
After entering the code on the right, after a brief loading period, 
the probability that the code was generated by an LLM or written by 
a human will be displayed. The percentage bar changes colour 
depending on the percentage. On the same page, the reasons why an LLM code detector 
is useful are quickly listed.


%%%%%%%%%%%%%%%%%
In the dataset section, every dataset available from Hugging 
Face is automatically downloaded. The purposes for which the 
dataset is recommended for use are provided (for example, AIG 
is recommended for testing the detection method's ability to 
identify competitive code).

Users are also given the option to decide the size of the 
subset they wish to obtain from the dataset. A balanced subset 
is then automatically created based on: LLM generators, 
programming languages, and code correctness. When these 
fields are not present, they are ignored. Additionally, a 
graph is displayed to visually represent the average length 
of code without comments.


%%%%%%%%%%%%%%%%%
%\clearpage
\section{Implementation and Features}